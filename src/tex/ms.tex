%Define document class
\documentclass[twocolumn]{aastex631}
\usepackage{showyourwork}
\newcommand{\code}[0]{\texttt{SilkScreen}}
\newcommand{\artpop}[0]{\texttt{ArtPop}}
\newcommand{\sbi}[0]{\texttt{sbi}}

% Begin!
\begin{document}

% Title
\title{Constraining the Distances and Physical Properties of Semiresolved Galaxies with Simulation Based Inference}

\author[0000-0001-8367-6265]{Tim B. Miller}
\affiliation{Center for Interdisciplinary Exploration and Research in Astrophysics (CIERA), Northwestern University,1800 Sherman Ave, Evanston, IL 60201, USA}
\affiliation{Department of Astronomy, Yale University, New Haven, CT 06511}

\author{Imad Pasha}
\affiliation{Department of Astronomy, Yale University, New Haven, CT 06511}

\author[0000-0002-5283-933X]{Ava Polzin}
\affiliation{Department of Astronomy, Yale University, New Haven, CT 06511}

\author{Pieter G. van Dokkum}
\affiliation{Department of Astronomy, Yale University, New Haven, CT 06511}

%\author{Shany Danieli}??
%\author{Johnny Greco}??

\begin{abstract}
%% Re-write
We present the \code{} package, which can accurately recover properties of globular clusters and near-field dwarf galaxies ($D\lesssim 20$ Mpc), including stellar mass, distance, and metallicity directly from imaging from various wide field surveys. \code{} includes models created via simulation-based inference (sbi), which trains a neural network to directly estimate the posterior that maps these properties, as well as details of the survey (e.g., PSF, exposure depth), to galaxy images and estimates the uncertainties in this mapping. We train this network using images simulated by the \artpop{} package, which can create realistic, physically-motivated images of dwarf galaxies and inject them into sky images, for several widely-used surveys. Once trained on images spanning the range of expected properties, this network can sample from the estimated posterior to make predictions for new galaxy inputs in a matter of seconds without ever calling \artpop{}, a vast improvement over the tens-to-hundreds of hours required to perform a classical Bayesian fit to the same data using MCMC. Furthermore, the method allows the maximal extraction of resolved, 2D information about each galaxy --- information which must be represented by summary statistics when using traditional methods. In this first paper, we present the methodology and test metrics, and provide the package publicly. 

\end{abstract}

\keywords{Machine Learning}



\section{Introduction} \label{sec:intro}

The study of dwarf galaxies, loosely defined here as galaxies with stellar mass $10^5<M_{*}<10^9$ $M_{\odot}$, represents an exciting avenue to study the physics of galaxy formation and dark matter. They are typically more dominated by dark matter and have fewer baryonic processes than a typical $L_{*}$ galaxy, making them useful sites to distinguish varying dark matter models. A primary challenge to such studies is the accumulation of local samples of such galaxies with known distances (enabling further studies of intrinsic properties, such as mass). While the surface brightness depths of previous surveys have limited the joint mass-distance discovery space of dwarf galaxies, upcoming surveys, including the Legacy Survey of Space and Time (LSST) at Vera Ruben Observatory and those that will be carried out by the Roman Space Telescope bring the promise of many new identified ``fuzzy blobs'' in the local universe. 

These discoveries will enable a new era of dwarf galaxy studies, but will be initially limited by the fact that these galaxies will have unknown distances, masses, and metallicities. While the latter two can be measured via (expensive) spectroscopy, the distance to these systems will remain unconstrained. Current methods for deriving these distances include surface brightness fluctuations (SBF) and tip of the red giant branch (TRGB) measurements. The latter is only possible with resolved RBG stars, meaning, in general, that high-resolution space-based data must be gathered (with, e.g., \textit{HST} or \textit{JWST}), making the follow up of large numbers of candidate galaxies a challenge. SBF can be carried out using the wide-field discovery data, but relies on a selection of metallicity and cannot marginalize over joint uncertainties in mass and metallicity when determining distances.

An additionally, as-yet unexplored avenue for constraining the critical properties of distance, mass, and metallicity for wide-field dwarf galaxy candidates is the use of forward modeling. Forward modeling; i.e., generating realistic models from the parameters of interested and comparing these models to the data to determine a best-fit (or posterior space) has been successfully employed in a variety of astrophysical contexts (cite). Forward modeling is, at current, most often paired with Bayesian Frameworks in order to find not only singular best fitting models, but parametrize realistic uncertainties on the model parameter of interest (and to marginalize over those which are not of interest).

Bayesian inference requires the imposition of priors on the parameters of interest (though these need not be restrictive), and has the advantage of producing trustworthy uncertainties on the parameters of interest --- a trait lacking in simple grid-based $\chi^2$ minimization schemes. Bayesian inference relies on Bayes theorem, which states that 
\begin{equation}\label{bayes}
    p(\theta|x) = \frac{p(x|\theta)p(\theta)}{p(x)}.
\end{equation}
In this equation, $p(\theta|x)$ is known as the \textit{posterior} distribution, and represents the quantity of interest: the probability of some vector of model parameters $\theta$ given the observed data $x$ on hand. By finding the $\theta$ which maximizes this quantity, one can find the model that ``best" fits the data. On the right hand side, $p(x|\theta)$ is known as the \textit{likelihood}, and describes the probability of our data given some vector of parameters $x$. The likelihood can be parameterized in different ways; a simple likelihood might just be the measured $\chi^2$ between a model and data (i.e., an assumption that the likelihood is Gaussian). $p(\theta)$ is the \textit{prior}, and represents the distributions from which vectors of $\theta$ will be drawn. A simple prior might be a top-hat (all values of $\theta$ equally likely between some bounds) or a Gaussian with a central value and standard deviation. In the denominator, $p(x)$ is known as the \textit{evidence}, and is generally challenging to compute --- some Bayesian methods do so, others do not, as it is a normalizing constant and the left hand side need only be maximized. Because the posterior is a distribution, the multi-dimensional surface around the maximally likely vector of $\theta$ can be used to estimate the uncertainty in the parameters in a robust way. 

The code Prospector, for example, uses the FSPS spectral synthesis code to generate realistic models of galaxy spectral energy distributions, then compares these models to gathered photometry and spectroscopy in a Bayesian framework using MCMC to sample the posterior space of the model parameters of interest (e.g., stellar mass, metallicity, star formation rate and history), while marginalizing over nuisance parameters. 

Such Bayesian frameworks, which have become ubiquitous across the last decade, have generally focused on a certain class of inference problem in which an explicit likelihood function can be written down. In the case of a galaxy SED fit, this might be a chi-squared measure between the model photometry and the data photometry. An alternative class of inference problem has defied such frameworks: those for which an explicit likelihood cannot be written. 

The problem of inferring the distance, mass, and metallicity of unknown ``fuzzy blobs'' in wide field imaging is one such problem. However, tools currently exist which can forward model the appearance of such galaxies in their respective surveys. In particular, the \artpop{} code forward models stellar systems by drawing a stellar population from an IMF, placing stars spatially according to a given model (e.g., Sérsic), convolving with an observational PSF, injecting into realistic backgrounds, and then simulating observations for any telescope and observing conditions (see section BLANK for a more detailed summary). In short, for the stellar populations \artpop{} is designed to create, it can create a final image theoretically indistinguishable from one taken by the respective telescope of an object of those properties. 

While this tool is powerful in its own right, it has, to date, not been used to actually \textit{fit} the properties of systems detected observationally. The reason, as described above, is that both the data and models (i.e., 2D images) represent stochastic draws from some distribution of systems for which the underlying properties are the same. In this case, writing down an explicit likelihood function (e.g., the chi-squared between the pixels of a model image and a data image) is impossible (or nonsensical). Another way of stating this is that while tools like an SED simulator (like FSPS) have \textit{deterministic} outputs (and the data, correspondingly, has only noise associated with measurement), in the case of semi-resolved galaxy imaging, this is not the case.

Simulation based inference is a novel and rapidly-developing suite of algorithms which leverage machine learning to directly ``learn" the posterior distribution not just of a single target with a single set of parameters, but for \textit{all} possible targets with \textit{all} possible parameters. This involves simulating many, many pairs of labels ($\theta$ vectors) and data (models), but only once. A machine learning model is then trained to learn the posterior, returning an object known as a \textit{neural density estimator} (NDE). When new data (i.e., real observations) are provided, the posterior can be directly sampled (extremely efficiently), providing, nominally, the same predictions and confidence intervals that would have been returned by an MCMC applied to the same data. 

This method has the benefit of amortizing the heavy computational load --- that is, the costly computation is computed once, ahead of time. It has been used recently in a variety of astrophysics and physics contexts (cite). A key requirement of these methods is that the physical models created are nearly indistinguishable from the real data being used, and in particular, that the noise properties of the models are consistent with what is seen in the data. Thankfully, \artpop{} is well-suited to this task, producing mock observations which are expected to be extremely accurate to real data (cite). 

A key advantage of this method for the use of \artpop{} in inference problems is that during the simulation stage, many instances of models will be created from the same input vector of parameters. By using a convolutional neural network (CNN) as an embedding network to perform the training directly on these images, we extract the maximal amount of information, yet additionally train the network to ignore the stochasticity present across models with the same input parameters. The generated summary statistics from this step are thus generated by (ideally) only the features in the image which distinguish models with \textit{different} parameters (i.e., different distances). 

By directly learning the posterior, it can be sampled from rapidly. Once a set of simulations has been used to train the neural network and create the neural density estimator, real data can be input and posterior samples for the model parameters can be drawn in a matter of seconds. So long as the data and simulated data are, indeed, close enough in nature, one retrieves a fully Bayesian estimate of the model parameters. 

It should be noted that, in adopting this framework for the problem of galaxy image inference, two primary avenues are available. We deem these the ``bespoke'' approach and the ``one model to rule them all'' approach. In the former, all three steps --- simulation, training, and sampling --- are carried out for individual targets. As such, the benefits of amortization are lost, but in contrast, models are only created which match all known properties of the data, e.g., PSF, position angle, Sérsic index. As a result, no simulations are ``wasted'' by being far from the true posterior for a given data image, and the constraints on the model parameters should be commensurately tighter. 

Using a ``bespoke'' model tuned to the particular galaxy being fit imposes several requirements in this framework: it is time consuming, and requires the use of graphical processing units (GPUs) to support the training of the CNN (simulations need not be carried out on GPU). The feasibility of such an approach has improved dramatically in the past several years, particularly with the advent of cloud-based computing for which GPU access can be purchased relatively inexpensively.

On the other hand, a ``one model to rule them all'' approach involves generating orders of magnitude more simulations in order to train the NDE to perform inferences on a much wider array of input data. This has the benefits of amortization --- the NDE could be trained on a supercomputer using many millions of simulations once, then used on any input data appropriately matched (e.g., right survey, for which images were taken with the same telescope under similar observing conditions). In this limit, all new images can be fit in a manner of seconds, though as a large portion of the posterior space would be ignored, the posteriors on individual parameters may be less constrained. 

It worth noting that at no point is the task of model creation being entrusted to an neural network; rather, the relationship between model parameters and the generated images themselves is learned. 

\section{The \code{} Framework}
\subsection{Simulating images Dwarf Galaxies}
\label{sec:sfh}
To simulate dwarf galaxies we utilize the \artpop{} pacakge. \artpop{} generates artificial images of dwarf galaxies ``from the ground up" by drawing stellar masses from an IMF, calculates luminosities using isochrones, draws positions for each start using a pre-defined brightness profile and then directly injects stars into an image before convolving with the PSF.

\artpop is designed to be highly modular and flexible, meaning many types of systems can be generated but also many choices that must be made. For all the simulations shown here we use a Kroupa IMF (ref) and MIST  Isochrones simulated with rotation $v/v_{\rm crit} = 0.4$ (ref). In \artpop{} we specify a magnitude limit (using the \texttt{mag\_limit} keyword) for which to resolve individual stars. This saves massively on computing resources by rendering the many low luminosity stars as a smooth background. For our case the SBF signal is always dominated by bright giant stars (REF).  Unless otherwise specified we set the cutoff in the $r$ band at $27\ mag$ (XXX CHECK XXX).

The next important set of choices is how to parameterize the star formation and metalicity history. For this initial exploration we opt for a simple parameterization: As series of simple stellar populations with a fixed metalicity. The parameters and priors of this ``Default'' Dwarf galaxy model are summarized in Table~\ref{tab:dwarf_model}. These choice were informed by the analysis of SBF magnitude as a function of age presented in Greco et al. 2021 (REF). Particularly we wanted to allow for more flexibility for the younger population where the SBF magnitude varies quickly as function of age. For metallicity assume a mass-metallicity prior following Kirby et. al (REF) and keep it fixed between the 3 stellar populations. While this is not fully realistic, as the metallicity is known to increase with successive populations of stars, we opt for simplicity in this initial exploration.

In order to produce mock images with realistic noise profiles we inject them into real data. Using \artpop{} we first render pristine, i.e. noise free, images that have been convolved to the PSF and pixel scale of the specified survey. Then we add Poisson noise before injecting these images into patches of real data. We do not pre-select "empty" regions as we want to pass the full distribution of data quality to the network such that it can produce realistic uncertainties.

\begin{table*}
    \centering
    \caption{The parameters are priors for our default dwarf model parameterized as 3 SSPs with a fixed metallicity}
    \small
    \begin{tabular}{c c c} 
        Parameter & Description & Prior \\  \hline \hline
         D & Distance to galaxy (Mpc) & Uniform: $D_{\rm min}$ -  $D_{\rm max}$ (User Specified) \\
         $\log (M_*/M_\odot)$ & Total Stellar Mass & Log-uniform: $M_{\rm*, min}$ -  $M_{\rm* , max}$ (User Specified) \\
         $Z$ & Metallicity parameterized as [Fe/H] & Gaussian following Kirby et al. MZR with $\sigma = 0.255$ \\
         $F_{y}$& Fraction of mass in the young population & Trunc. Norm.: $\mu=0$, $\sigma=0.05$ from $0-0.2$\\
         $Age_{y}$& Age of the the young population (100 Myr) & Uniform: $0.5-5$\\
         $F_{m}$& Fraction of mass in the medium population (Age = 2 Gyr) & Trunc. Norm.: $\mu=0.4$, $\sigma=0.2$ from $0-0.8$\\
         $F_{o}$& Fraction of mass in the old population (Age = 12 Gyr) & Dependent on $F_{y}$ and $F_{m}$: 1 - $F_{y}$ - $F_{m}$\\ \hline  
    \end{tabular}
    \label{tab:dwarf_model}
\end{table*}

\subsection{Network Architecture}
\label{sec:NN}
To perform simulation based inference we use the \sbi{} python package {REF}. We use the Sequential Neural Posterior Estimate (SNPE) - C methodology to map data directly to the posterior. The network we us contains two components, an embedding network to extract summary statistics directly from images that is trained simultaneously to the normalizing flow model to predict the posterior distribution. 

A normalizing flow is a series of learned bijective transforms which map a simple distribution (Often a multivariate normal) to any complex distribution. In this way normalizing flows ``learn'' any complex distributions, optionally conditioned on input parameters. These transforms are designed to be easily invertable and have tractable Jacobians such that can be be sampled from and trained easily. In simulation based inference, it is used to estimate the posterior distribution, given a set of input data. For \code{} we use a Neural Spline Flow (NSF) (REF) consisting of five transformations with 50 hidden features each, eight spline bins and dropout probability of 0.2 in the hidden layers.

In simulation based inference it is common to use a set of summary statistics to describe the data. However in our case of inferring properties of dwarf galaxies it is unclear what the best summary statistics are or how to reliably extract them. As our input data is highly complex we combine the flow with and embedding network, that is trained alongside the flow, to distill the input images to a set of learned summary statistics.

For this we use a Residual Neural networks (ResNets) (REF). Compared to traditional Convolutional Neural Networks (CNNs), ResNets include skip connections where the output of  a layer is added to the input. Compared to CNNs, they have veeb shown to be easier to optimize and allow for deeper networks without degradation. Our implementation is based on that in \texttt{torchvision}. It contains an initial convolution layer which maps any number of input filters to 16 channels proceeded by four segments. Each segments contains two bottleneck residual blocks with an increasing number of channels, 16,32 64 and 128 respectively. This is followed by an average pooling to 512 features and a fully connected layer resulting in the desired number of summary statistics.

\subsection{Current Pipeline: Bespoke Simulations and Training}
\label{sec:walkthrough}

The current pipeline for inference with \code{} is tailored to each specific galaxy of interest. A number of observations need to be assembled. First and foremost cutouts of the galaxy in a set of filters, along with their PSFs. Then details about the telescope and survey like the average sky surface-brightness and exposure time ot accurately calculate the noise properties. Finally an estimate of the observed light distribution, the current supported parameterizations are Plummer or Sersic Profiles. This is all collected into a \texttt{SilkScreenObservation} class which can then be used for inference. 

Next one must choose a parameterization for the star-formation history, and prior for the parameters. We provide routines to generate the default options discussed in section~\ref{sec:sfh} in the \texttt{silkscreen.simmers} and \texttt{silkscreen.priors} scripts respectively. While we have only started with these few defaults, since \artpop{} is highly modular and flexible, so is \code{} and we plan to add more options in the future.

Then training the posterior estimation network is performed using the \sbi{} python package. Unless otherwise specified we train the network in three rounds restricting the prior each time following the Truncated Sequential Neural Posterior Estimation (TSNPE) method. (REF) This methodology uses the posterior of each successive round to truncate the prior and achieve more targeted simulations for a given observation. Unlike previous sequential methods it allows for simpler training and loss function by truncating the prior rather than having to compare to previous iterations of the posterior. Through experimentation we found including half the total simulation budget in the first round, drawn from the prior, and the splitting the other half among two successive rounds was successful. Unless otherwise specified in this paper we train the posterior estimator in three round with 50k, 25k and 25k simulations respectively.

For training we use the AdamW optimizer and apply a learning rate of $2\times 10^{-4}$ for the flow part of the network and $1\times 10^{-6}$ to the embedding part of the posterior estimator network with a batch size of 150. After the initial round of training we lower each learning rate by a factor of 0.1. The weight decay is set to $10^{-4}$ for both sections of the network and we set the number of learned summary statistics to 16. Each round of training continues until the loss of a validation set (10\% of total data) fails to improve for 15 successive epochs. Before training the images are passed through arc-sinh normalization and then z-scoring, performed overall on all filters simultaneously to preserve color information. Similarly the parameters used to generate the simulations are z-scored. (XXX Double check all of these are correct XXX)

To optimize all these hyper-parameters we use the \texttt{optuna} (REF) package applied to the test case shown in section~\ref{self-test}. As with many networks, we find the batch size and learning rates to be the most impactful hyperparameters. In particular we found de-coupling the learning rates of the flow and embedding network was crucial for reliable training. We discuss the choices made in the architecture of the network and hyperparameters in section\ref{sec:issues}.

\section{Inference on a Mock Galaxy}
\label{self-test}
Talk about performing inference on a simulated galaxy. Show recovered posteriors along with SBC benchmarks showing a well behaved posterior

\section{Example Applications}
\subsection{Halo Globular Clusters}
\label{sec:gc}
As a first application to real data we apply \code{} to two halo globular clusters (GCs). They represent a useful case for \code{} as they are thought to be a relatively simple stellar populations, single aged and old and have well measured properties through alternative methods. We choose two GCs to test on, Palomar 4 and Palomar 13 for which we will be fitting to DeCALS survey data. These were choosen as they are both well studied GCs, lie within DeCALS, roughly follow Plummer profiles and crucially lie at vastly different differences. Palomar 14 has a known distance of roughly 100 kpc compared to Palomar 13 which lies much closer, 25 kpc from the sun. (REF) The difference between these two creates an ideal test case to ensure \code{} can infer different distances while all of the other variables (e.g. number of bands used, integration time, stellar population etc.) are kept constant.

We use largely the same procedure as described in Section. \ref{sec:walkthrough} with a few minor tweaks. We use a single stellar population with uniform priors in distance, $\log M_*$, metallicity and age summarized in Table~\ref{tab:ssp_model}. Additionally we find a lower initial learning rate for the normalizing flow is warranted here, otherwise the training loss becomes very stochastic. We start at $2\times 10^{-5}$ for the initial round with the rest of the procedure following what is described in section~\ref{sec:NN}

\begin{table*}
    \centering
    \caption{The parameters are priors for our SSP model parameterized used to model globular clusters}
    \small
    \begin{tabular}{c c c} 
        Parameter & Description & Prior \\  \hline \hline
         D & Distance (kpc) & Uniform: $D_{\rm min}$ -  $D_{\rm max}$ (User Specified) \\
         $\log (M_*/M_\odot)$ & Total Stellar Mass & Log-uniform: $M_{\rm*, min}$ -  $M_{\rm* , max}$ (User Specified) \\
         $Z$ & Metallicity parameterized as [Fe/H] & Uniform: $-2.25-0.25$ \\
         $Age$& Age of the the SSP (Gyr) & Uniform: $5-12$\\ \hline
    \end{tabular}
    \label{tab:ssp_model}
\end{table*}

\begin{figure*}
    \centering
    \includegraphics[width=0.45\textwidth]{figures/Pal4_post_corner.pdf}
    \includegraphics[width=0.45\textwidth]{figures/Pal13_post_corner.pdf}
    \includegraphics[width=0.4\textwidth]{figures/Pal4_ims_post.pdf}
    \hspace{1cm}
    \includegraphics[width=0.4\textwidth]{figures/Pal13_ims_post.pdf}
    \caption{Caption}
    \script{plot_gc.py}
    \label{fig:GCs}
\end{figure*}


\subsection{Dwarfs in the Sculptor Group}
\label{sec:scul}

\begin{figure*}
    \centering
    \includegraphics[width=0.65\textwidth]{figures/ESO_294_10_post_corner.pdf}
    \includegraphics[width=0.6\textwidth]{figures/ESO_294_10_ims_post.pdf}
    \caption{Caption}
    \script{plot_scul_dw.ipynb}
    \label{fig:GCs}
\end{figure*}

\subsection{Galaxies in HSC}
\label{sec:hsc}

\section{Current Limitations and Pitfalls}
\label{sec:issues}
Oh so many things to discuss here: Paramterization of SFH, difficulty in training and producing well-calibrated posteriors, large computational cost of 'bespoke' galaxy models.

\section{Future Outlook}
\label{sec:future}
First is a systematic benchmark of distances/stellar population properties against a known set of galaxies. Taking advantage of the amortization possibilities to build 'one model to rule them all' applicable to a whole survey etc. Honing in on stellar population properties of known galaxies with lots of data.

\section{Summary}
\label{sec:summary}
Really cool idea that I think will be super useful but a lot of work left to do.

\section*{Acknowledgements}
The authors would like to thank Aritra Ghosh for valuable discussions on the design, implementation and training of the Neural Posterior Estimator. TBM was supported by a CIERA Postdoctoral Fellowship.

The Legacy Surveys consist of three individual and complementary projects: the Dark Energy Camera Legacy Survey (DECaLS; Proposal ID \#2014B-0404; PIs: David Schlegel and Arjun Dey), the Beijing-Arizona Sky Survey (BASS; NOAO Prop. ID \#2015A-0801; PIs: Zhou Xu and Xiaohui Fan), and the Mayall z-band Legacy Survey (MzLS; Prop. ID \#2016A-0453; PI: Arjun Dey). DECaLS, BASS and MzLS together include data obtained, respectively, at the Blanco telescope, Cerro Tololo Inter-American Observatory, NSF’s NOIRLab; the Bok telescope, Steward Observatory, University of Arizona; and the Mayall telescope, Kitt Peak National Observatory, NOIRLab. Pipeline processing and analyses of the data were supported by NOIRLab and the Lawrence Berkeley National Laboratory (LBNL). The Legacy Surveys project is honored to be permitted to conduct astronomical research on Iolkam Du’ag (Kitt Peak), a mountain with particular significance to the Tohono O’odham Nation.

NOIRLab is operated by the Association of Universities for Research in Astronomy (AURA) under a cooperative agreement with the National Science Foundation. LBNL is managed by the Regents of the University of California under contract to the U.S. Department of Energy.

This project used data obtained with the Dark Energy Camera (DECam), which was constructed by the Dark Energy Survey (DES) collaboration. Funding for the DES Projects has been provided by the U.S. Department of Energy, the U.S. National Science Foundation, the Ministry of Science and Education of Spain, the Science and Technology Facilities Council of the United Kingdom, the Higher Education Funding Council for England, the National Center for Supercomputing Applications at the University of Illinois at Urbana-Champaign, the Kavli Institute of Cosmological Physics at the University of Chicago, Center for Cosmology and Astro-Particle Physics at the Ohio State University, the Mitchell Institute for Fundamental Physics and Astronomy at Texas A\&M University, Financiadora de Estudos e Projetos, Fundacao Carlos Chagas Filho de Amparo, Financiadora de Estudos e Projetos, Fundacao Carlos Chagas Filho de Amparo a Pesquisa do Estado do Rio de Janeiro, Conselho Nacional de Desenvolvimento Cientifico e Tecnologico and the Ministerio da Ciencia, Tecnologia e Inovacao, the Deutsche Forschungsgemeinschaft and the Collaborating Institutions in the Dark Energy Survey. The Collaborating Institutions are Argonne National Laboratory, the University of California at Santa Cruz, the University of Cambridge, Centro de Investigaciones Energeticas, Medioambientales y Tecnologicas-Madrid, the University of Chicago, University College London, the DES-Brazil Consortium, the University of Edinburgh, the Eidgenossische Technische Hochschule (ETH) Zurich, Fermi National Accelerator Laboratory, the University of Illinois at Urbana-Champaign, the Institut de Ciencies de l’Espai (IEEC/CSIC), the Institut de Fisica d’Altes Energies, Lawrence Berkeley National Laboratory, the Ludwig Maximilians Universitat Munchen and the associated Excellence Cluster Universe, the University of Michigan, NSF’s NOIRLab, the University of Nottingham, the Ohio State University, the University of Pennsylvania, the University of Portsmouth, SLAC National Accelerator Laboratory, Stanford University, the University of Sussex, and Texas A\&M University.

BASS is a key project of the Telescope Access Program (TAP), which has been funded by the National Astronomical Observatories of China, the Chinese Academy of Sciences (the Strategic Priority Research Program “The Emergence of Cosmological Structures” Grant \# XDB09000000), and the Special Fund for Astronomy from the Ministry of Finance. The BASS is also supported by the External Cooperation Program of Chinese Academy of Sciences (Grant \# 114A11KYSB20160057), and Chinese National Natural Science Foundation (Grant \# 12120101003, \# 11433005).

The Legacy Survey team makes use of data products from the Near-Earth Object Wide-field Infrared Survey Explorer (NEOWISE), which is a project of the Jet Propulsion Laboratory/California Institute of Technology. NEOWISE is funded by the National Aeronautics and Space Administration.

The Legacy Surveys imaging of the DESI footprint is supported by the Director, Office of Science, Office of High Energy Physics of the U.S. Department of Energy under Contract No. DE-AC02-05CH1123, by the National Energy Research Scientific Computing Center, a DOE Office of Science User Facility under the same contract; and by the U.S. National Science Foundation, Division of Astronomical Sciences under Contract No. AST-0950945 to NOAO.

\bibliography{sample631}{}
\bibliographystyle{aasjournal}


\end{document}

% End of file `sample631.tex'.

\section{Artpop}
\artpop{} is a pure-Python package developed to produce realistic images of dwarf galaxies as they would appear in various observational surveys. The code uses a modular structure to simulate the distribution of stars for systems of desired properties (e.g., mass and metallicity), and ``place them" in a survey (e.g., Hypersuprime-cam SSP) using the survey mirror aperture and focal length, exposure time, and sky brightness. This galaxy image with added noise can be injected directly into real images of blank (or less blank) sky from these surveys. 
\subsection{Simple Stellar Populations}
\subsubsection{Composite Stellar Populations}
\subsection{Galaxy Properties}
\subsubsection{Stellar Mass}
\artpop{} allows the user to specify the number of stars to simulate, or a total mass, and draws a sample of stars from a Chabrier (2003) IMF (true?? \textcolor{red}{[Ava: I believe the default IMF is Kroupa. We could say here that ArtPop allows users to choose between Chabrier, Kroupa, and Salpeter IMFs, and then either say that we've chosen to use the default or give a rationale for choosing a different option.]}); here, we work primarily with stellar mass. When a magnitude limit is imposed, \artpop{} only draws individual stars for the fractional mass that should exist above the limit, and simulates the rest of the light of the galaxy as a smooth profile (e.g., Sersic \textcolor{red}{[Ava: Since there are only three built-in options for galaxy profiles I think we could follow whatever convention we choose for the IMF and potentially list S\'{e}rsic, Plummer, and Uniform and then state our choice.]}). We select magnitude limits in this work based on the surveys being used. 
\subsubsection{Metallicity}
Because \artpop{} is built on generating either single simple stellar populations (SSPs) or composite stellar populations (CSPs) comprised of multiple SSPs, users specify a single [Fe/H] for each SSP in their model. Stars are then drawn from isochrones at that metallicity (potentially interpolating where necessary).

\subsection{Survey Properties \& Noise}
Once an ideal galaxy image has been created, \artpop{} passes it through a module which applies the observing conditions, including sky brightness and PSF, and determining the size the galaxy should appear on the detector of the instrument being used to observe it based on its distance. 

\subsection{Joint Priors}
\code{} allows for the implementation of joint priors (that is, priors for which one parameter depends on the value of another parameter). In \code{}, we provide an optional, built in, joint prior based on the Mass-Metallicity Relation (MZR; \citealt{2005MNRAS.362...41G, 2013ApJ...779..102K}), but also demonstrate how one can implement any joint prior of choice. 

\section{Silkscreen}
Silkscreen\footnote{This name is a reference to the silkscreen process used by Andy Warhol to define the Pop Art movement.} combines \artpop{} for model generation with the SBI framework \sbi{}, implementing a custom convolutional neural network for the feature extraction in the simulation stage. Users set or select priors on the model properties (mass, distance, metallicity, position angle, Sérsic index, etc.), choosing tight priors for known quantities and flat priors for unknown properties. 